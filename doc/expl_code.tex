\section{Aufbau Diagramm-Element}
\subsection{Joint-Group}
Eine Joint-Group ist die Implementierte Art und Weise wie gleichartige Joints gruppiert werden.
Eine solche Gruppe besitzt Maximal- bzw. Minimalzahlen von Joints, wobei neue Joints durch eine
Factory erstellt werden. Alle diese Informationen werden beim Konstruktoraufruf übergeben.
Elemente wie die Operation haben beispielsweise view Joint-Groups. Zwei für Flow- bzw. Abhängigkeits-Eingang
und zwei für Flow- bzw. Abhängigkeits-Ausgang.

\subsection{Joint}
Ein Joint ist ein Verbindungsknoten. Je nach Konfiguration kann ein Joint als Eingang, Ausgang oder beides dienen.
Bisher gibt es sowohl einen \textit{FlowJoint} als auch einen \textit{DependencyJoint}.

\pagebreak
\section{Erweiterung}
Beim Architekturentwurf wurde darauf geachtet das Programm so einfach wie möglich erweiterbar zu machen.
Dieses Kapitel zeigt wie der vorhandene Code erweitert und um Inhalt ergänzt werden kann.

\subsection{Erweiterung um einen Diagrammtyp}
Im Folgenden soll ein neuer Diagramm-Typ mit dem Namen ''Example'' beispielhaft erstellt werden.

\subsubsection{Datenmodell}
Für alle bisherigen Diagramme wurde ein eigenes Paket erstellt. Das ist keine Voraussetzung, hilft allerdings bei der
Strukturierung. Im Folgenden wird deshalb davon ausgegangen, dass das Paket \textit{com.tallbyte.flowdesign.data.example}
verwendet wird.
Hier muss eine Klasse ''ExampleDiagram'' erstellt werden, welche von \textit{Diagram} erbt. Sollte es erwünscht sein,
dass Elemente des Diagramms von einer bestimmten Art sind, so kann dies über den generischen Parameter von 
\textit{Diagram} angegeben werden.

\begin{figure}[h!]
	\centering
	\begin{lstlisting}
public class ExampleDiagram extends Diagram<DiagramElement> {

    public ExampleDiagram(String name) {
        super(name, null);
    }

    public EnvironmentDiagram(String name, DiagramElement root) {
        super(name, root);
    }
}
	\end{lstlisting}
	\caption{Beispiel Diagramm-Klasse}
\end{figure}

\subsubsection{View}
Das Diagramm hat keine eigentliche View. Diese Aufgabe wird direkt von der Klasse \textit{DiagramPane} im Modul 
\textit{javafx} umgesetzt. Hier ist Diagramm-spezifisch nichts zu verändern.

\subsubsection{Handler}
Der \textit{DiagramHandler} ist die Diagramm-spezifische Schnittstelle, die Aufgaben wie die Erstellung von neuen
Diagramm-Instanzen oder die Bereitstellung verfügbarer Properties übernimmt. Zur Vereinfachung gibt es bereits eine
abstrakte Klasse, die die komplexesten Aspekte des Intefaces \textit{DiagramHandler} verbirgt. Diese kann erweitert
werden womit folgende Klasse im Kontext des Beispiels zu erstellen ist.

\begin{figure}[h!]
	\centering
	\begin{lstlisting}
public class ExampleDiagramHandler extends DiagramHandlerBase<ExampleDiagram, DiagramElement, DiagramImage> {

    public ExampleDiagram() {
        addEntries("System", System.class,
                System::new,
                EllipseDiagramImage::new,
                SystemElementNode::new
        );
    }

    @Override
    protected ExampleDiagram createNewDiagramInstance(String name) {
        return new ExampleDiagram(name);
    }
    
    @Override
    public ObservableList<Property<?>> getDiagramProperties(ExampleDiagram diagram) {
        ObservableList<Property<?>> list = super.getDiagramProperties(diagram);

		// add properties

        return list;
    }
}
	\end{lstlisting}
	\label{diagram_handler}
	\caption{Beispiel Diagramm-Handler-Klasse}
\end{figure}
Der Methodenaufruf von \textit{addEntries} wird später noch im Kapitel zum Hinzufügen neuer Diagramm-Element
erklärt (\ref{add_element}). Ansonsten kann in der Methode \textit{getDiagramProperties} eine Liste mit verfügbaren
Properties für ein Diagramm erstellt werden. Diese Liste wird verwendet, um auf der Oberfläche Optionen für das
jeweilige Diagramm einzublenden.
\\r Diagrammtyp
\\
Damit der Handler auch aktiv genutzt wird, muss noch ein Methodenaufruf in der Klasse \textit{DiagramHandler} gemacht
werden.

\begin{figure}[h!]
	\centering
	\begin{lstlisting}
static {
    addHandler(EnvironmentDiagram.class, new EnvironmentDiagramHandler());
    addHandler(FlowDiagram.class, new FlowDiagramHandler());
    addHandler(MaskDiagram.class, new MaskDiagramHandler());
        
    // diese Zeile muss eingefuegt werden
    addHandler(ExampleDiagram.class, new ExampleDiagramHandler());
}
	\end{lstlisting}
	\label{diagram_handler}
	\caption{Diagramm-Handler verfügbar machen}
\end{figure}
\subsubsection{Serialisierung}
Damit der neu-erstellte Diagramm-Typ nun auch gespeichert werden kann muss ein (XML-)Serializer erstellt werden.
\\
// TODO Michael

\subsubsection{Strings}
Da der Projektbaum und andere UI-Elemente automatisch für alle verfügbaren Diagrammtypen erstellt werden müssen
die angezeigten Texte extern verwaltet werden. Hierfür wird das Java-eigene Ressourcensystem verwendet. Im Ressourcen-
Verzeichnis in der Maven-Struktur befindet sich das Resource-Bundle \textit{MessagesBundle}. Hier müssen folgende Strings
bereitgestellt werden:

\begin{itemize}
	\item tree.overview.ExampleDiagram = Example
	\item menu.edit.new.ExampleDiagram = New Example-Diagram...
	\item popup.new.ExampleDiagram.title = New Example Diagram
	\item popup.new.ExampleDiagram.field.name = Name
	\item context.new.ExampleDiagram=New...
\end{itemize}

\pagebreak
\subsection{Erweiterung um ein Diagramm-Element}
\label{add_element}
Im Folgenden soll ein neues Diagramm-Element mit dem Namen ''ExampleElement'' beispielhaft erstellt werden.

\subsubsection{Datenmodell}
// TODO Michael
Zuerst muss das entsprechende Datenmodell erstellt werden. Nach bisheriger Konvention gehören die Diagramm-Elemente
für ein spezielles Diagramm in das gleiche Paket wie die entsprechende Diagramm-Klasse. Das ist allerdings keine
Voraussetzung.

\begin{figure}[h!]
	\centering
	\begin{lstlisting}
public class ExampleElement extends DiagramElement {

    public static final String JOINT_GROUP = "io";

    public ExampleDiagramElement() {
    }

    @Override
    protected Iterable<JointGroup<?>> createJointGroups() {
        return new ArrayList<JointGroup<?>>() {{
            add(new JointGroup<>(System.this, JOINT_GROUP , 4, 4, element -> new DependencyJoint(element, JointType.INPUT_OUTPUT, 0, 0), 4));
        }};
    }

    public JointGroup<?> getJointGroup() {
        return getJointGroup(JOINT_GROUP);
    }

}
	\end{lstlisting}
	\label{diagram_handler}
	\caption{Beispiel Diagramm-Element-Klasse}
\end{figure}
Hier wird ein einfaches Element erstellt, dass eine einzige Joint-Gruppe bereitstellt, die minimal und maximal
vier Joints zur Verfügung stellt, wobei einzelne Joints sowohl als Eingang als auch als Ausgang verwendet werden
können. Zudem ist die Art der Verbindung, welche von den Joints erstellt werden kann eine Abhängigkeitsverbindung
(ein Kreis am Ziel).
\subsubsection{View}
Damit das neue Element später auch auf der Zeichenfläche angezeigt werden kann, muss erst eine View dafür erstellt
werden. Alle View müssen die Basis-Klasse \textit{ElementNode} erweitern. Diese stellt bereits einige wichtig
Funktionen wie das verteilen von Joints auf der Oberfläche oder das vergrößern / verkleinern zur Verfügung.

\begin{figure}[h!]
	\centering
	\begin{lstlisting}
public class ExampleElementNode extends ElementNode<DiagramImage> {

    private final ExampleElement example;

    public SystemElementNode(ExampleElement element, DiagramImage content) {
        super(element, content, Pos.CENTER);

        this.example = element;
    }

    @Override
    protected void setup() {
        super.setup();

        addJointsAcrossCircle(new JointGroupHandler(example.getJointGroup(), 0, 1));
    }
}
	\end{lstlisting}
	\label{diagram_handler}
	\caption{Beispiel Diagramm-Element-Node}
\end{figure}
Durch den Methodenaufruf \textit{addJointsAccrossCircle} werden alle Joints der Joint-Group des übergeben
ExampleElements komplett über den Umfang einer Ellipse verteilt, der sich durch Höhe und Breite des Elements
ergibt. Die \textit{0} steht hier für den Offset im Kreis (Intervall [0;1]), die \textit{1} steht für den
verwendeten Umfang (Intervall [0;1]) in Prozent.
\\
\\
Es stehen hier folgende Funktionen zur Verfügung:
\begin{itemize}
	\item addJointsAccrossRectangle
	\item addJointsAccrossRectangleCentered
	\item addJointsAccrossCircle
	\item addJointsAccrossCircleCentered
\end{itemize}
Der dritte Parameter beim Super-Konstruktoraufruf steht für die Position des Textfelds für den Namen. Gültige
Werte sind \textit{Pos.CENTER} und \textit{Pos.BOTTOM\_CENTER}.

\subsubsection{Image}
\subsubsection{Factory}
\subsubsection{Serialisierung}
// TODO Michael

\pagebreak
\subsection{Weitere Referenzen}
Auf Grund der Offenheit des Systems sind eine Vielzahl von Modifikationen möglich, die nicht alle in dieser
Dokumentation behandelt werden können. Allerdings sind bereits einige Beispiele im Programmcode vorhanden.
Die folgende Liste zeigt einige Referenzen für \textit{erweiterte} Funktionen:

\begin{itemize}
	\item Extra Textfeld im View (\textit{OperationalUnitElementNode\#setup()}
	\item Property-Verbindung mit dem Image (\textit{EndElementNode\#EndElementNode()})
\end{itemize}