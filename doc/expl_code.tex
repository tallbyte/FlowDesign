
\section{Erweiterung}
Beim Architekturentwurf wurde darauf geachtet das Programm so einfach wie möglich erweiterbar zu machen.
Dieses Kapitel zeigt wie der vorhandene Code erweitert und um Inhalt ergänzt werden kann.

\subsection{Erweiterung um einen Diagrammtyp}
Im Folgenden soll ein neuer Diagrammtyp mit dem Namen ''Example'' beispielhaft erstellt werden.

\subsubsection{Datenmodell}
Für alle bisherigen Diagramme wurde ein eigenes Paket erstellt. Das ist keine Voraussetzung, hilft allerdings bei der
Strukturierung. Im Folgenden wird deshalb davon ausgegangen, dass das Paket \textit{com.tallbyte.flowdesign.data.example}
verwendet wird.
Hier muss eine Klasse ''ExampleDiagram'' erstellt werden, welche von \textit{Diagram} erbt. Sollte es erwünscht sein,
dass Elemente des Diagramms von einer bestimmten Art sind, so kann dies über den generischen Parameter von 
\textit{Diagram} angegeben werden.


\subsubsection{View}
Das Diagramm hat keine eigentliche View. Diese Aufgabe wird direkt von der Klasse \textit{DiagramPane} im Modul 
\textit{javafx} umgesetzt. Hier ist Diagramm-spezifisch nichts zu verändern.

\subsubsection{Handler}
Der \textit{DiagramHandler} ist die Diagramm-spezifische Schnittstelle, die Aufgaben wie die Erstellung von neuen
Diagramm-Instanzen oder die Bereitstellung verfügbarer Properties übernimmt.

\subsubsection{Strings}
Da der Projektbaum und andere UI-Elemente automatisch für alle verfügbaren Diagrammtypen erstellt werden müssen
die angezeigten Texte extern verwaltet werden. Hierfür wird das Java-eigene Ressourcensystem verwendet. Im Ressourcen-
Verzeichnis in der Maven-Struktur befindet sich das Resource-Bundle ''MessagesBundle''. Hier müssen folgende Strings
bereitgestellt werden:

machsch du liste

\subsection{Erweiterung um ein Diagramm-Element}
\subsubsection{Datenmodell}
\subsubsection{View}
\subsubsection{Image}
\subsubsection{Factory}