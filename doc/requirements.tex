\section{Anforderungen}
Hervorgehend aus der Präsentation durch IT-Designers und unseren eignen Vorstellungen zum Projekt haben wir folgende Requirements ausgearbeitet:

\subsection{Funktionale Anforderungen}

\subsubsection{Diagramme}
Folgende Diagramme sollen dargestellt werden können:
\begin{itemize}
	\item System-Umwelt-Diagramme
	\item Maskenprototypen
	\item Flow Diagramme
\end{itemize}

\subsubsection{Plattformunabhängigkeit}
Durch Entwicklung in Java soll das Programm auf Windows, Mac OS und anderen Unix-Systemen lauffähig sein.

\subsubsection{Projektbasiertes Arbeiten}
Diagramme werden einem Flow-Projekt zugeordnet. Neue Diagramme können jederzeit hinzugefügt und entfernt werden.
Projekte solle einfach importiert und exportiert werden können bzw. Projekte sollen abgespeichert und geladen
werden können.

\subsubsection{Verlinkbarkeit}
Programmablauf für z.B. einen Button im Maskenprototyp sollen in den anderen Diagrammarten dargestellt werden können,
eine automatische Verlinkung soll stattfinden.

\subsubsection{Einhaltung von Verknüpfungslogiken}
Verlinkungen sollen automatisch auf Korrektheit überprüft werden. Ungültige Verbindungen, also zwischen inkompatiblen
Elementen, sollen blockiert werden.

\subsubsection{Codegenerierung (optional)}
Optional soll nach Erfüllung bereits genannter Requirements innerhalb des Projektplans versucht werden eine einfache
Codegenerierung aus Diagrammen zu implementieren.



\subsection{Nicht-Funktionale Anforderungen}

\subsubsection{Aussehen und Handhabung}
Die Oberfläche soll leicht verständlich und intuitiv bedienbar sein.

\subsubsection{Fehlertolleranz bzw. Fehlerabsicherung}
Wenn der Benutzter falsche bzw. ungültige Eingaben macht, so soll dieser darauf hingewiesen werden.

\subsubsection{Skalierbare Darstellung}
Die Anwendung soll sowohl bei Auflösungen wie 1024x768 als auch bei Auflösungen wie 1920x1080 korrekt
dargestellt werden. Zudem soll die Darstellung auf Retina-Displays bzw. im Allgemeinen bei aktivierter
Oberflächenskalierung fehlerfrei sein.
