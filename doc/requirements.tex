\section{Anforderungen}
Hervorgehend aus der Präsentation durch it-Designers und unseren eignen Vorstellungen zum Projekt haben wir folgende Requirements ausgearbeitet:

\subsection{Funktionale Anforderungen}

\subsubsection{Diagramme}
\begin{itemize}
	\item System-Umwelt-Diagramme
	\item Maskenprototypen
	\item Flow Diagramme
\end{itemize}

\subsubsection{Plattformunabhängigkeit}
\begin{itemize}
	\item Durch Entwicklung in Java auf Windows, macOS und anderen Unix-Systemen lauffähig
\end{itemize}

\subsubsection{Projektbasiertes Arbeiten}
\begin{itemize}
	\item Möglichkeit zum Erstellen von Diagrammen innerhalb eines Flow-Projektes, einfaches importieren/exportieren bzw. abspeichern von Projekten
\end{itemize}

\subsubsection{Verlinkbarkeit}
\begin{itemize}
	\item Programmablauf für z.B. einen Button im Maskenprototyp kann in den anderen Diagrammarten dargestellt werden und wird automatisch verlinkt 
\end{itemize}

\subsubsection{Einhaltung von Verknüpfungslogiken}
\begin{itemize}
	\item Automatische Überprüfung auf Korrektheit der Verknüpfungen wie etwa innerhalb eines System-Umwelt-Diagramms, Blockierung von falschen Verbindungen  
\end{itemize}

\subsubsection{Codegenerierung (optional)}
\begin{itemize}
	\item Bei Erfüllung oben genannter Requirements innerhalb des Projektplans: Implementierung von einfacher Codegenerierung aus den erstellten Diagrammen innerhalb eines Projektes
\end{itemize}
	
\subsection{Nicht-Funktionale Anforderungen}

\subsubsection{Aussehen und Handhabung}
Die Oberfläche soll leicht verständlich und intuitiv bedienbar sein.

\subsubsection{Fehlertolleranz bzw. Fehlerabsicherung}
Wenn der Benutzter falsche bzw. ungültige Eingaben macht, so soll dieser darauf hingewiesen werden.

\subsubsection{Skalierbare Darstellung}
Die Anwendung sollte soll sowohl bei Auflösungen wie 1024x768 als auch bei Auflösungen wie 1920x1080 korrekt
dargestellt werden. Zudem soll die Darstellung auf Retina-Displays bzw. im Allgemeinen bei aktivierter
Oberflächenskalierung fehlerfrei sein.
