\section{Requirements}
Hervorgehend aus der Präsentation durch it-Designers und unseren eignen Vorstellungen zum Projekt "Flow Design", haben wir folgende Requirements ausgearbeitet:

\begin{itemize}
	\item Möglichkeit zum erstellen von:
	\begin{itemize}
		\item System-Umwelt-Diagrammen
		\item Maskenprototypen
		\item Flow Diagrammen
	\end{itemize}

	\item Plattformunabhängigkeit
	\begin{itemize}
		\item Durch Entwicklung in Java auf Windows, OSX und anderen Unix-Systemen lauffähig
	\end{itemize}

	\item Projektbasiertes arbeiten
	\begin{itemize}
		\item Möglichkeit zum erstellen von Diagrammen innerhalb eines Flow-Projektes, einfaches importieren/exportieren bzw. abspeichern von Projekten
	\end{itemize}

	\item Verlinkung zwischen den einzelnen Diagrammen 
	\begin{itemize}
		\item Programmablauf für z.B. einen Button im Maskenprototyp kann in den anderen Diagrammarten dargestellt werden und wird automatisch verlinkt 
	\end{itemize}

	\item Beachtung von Verknüpfungslogiken durch das Programm
	\begin{itemize}
		\item Automatische Überprüfung auf Korrektheit der Verknüpfungen wie etwa innerhalb eines System-Umwelt-Diagramms, Blockierung von falschen Verbindungen  
	\end{itemize}

	\item Codegenerierung aus den Diagrammen (optional)
	\begin{itemize}
		\item Bei Erfüllung oben genannter Requirements  innerhalb des Projektplans: Implementierung von einfacher Codegenerierung aus den erstellten Diagrammen innerhalb eines Projektes
	\end{itemize}
\end{itemize} 