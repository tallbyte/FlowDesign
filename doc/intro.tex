
\section{Einleitung}

Für das Konstruieren benötigt jeder Ingenieur ein strukturiertes Vorgehen. \textFlowDesign{}
soll genau dort unterstützend wirken.
Zusammen mit der Firma IT-Designers GmbH hat sich unser Team im Wintersemester 2016/17 im
Rahmen des Projekts Softwaretechnik an die Entwicklung eines Programms, das diesen Ansatz
nutzt, gewagt.

\subsection{Motivation}
Wie erwähnt entstand das Projekt \textFlowDesign{} im Modul ''Projekt Softwaretechnik'' im vierten Semester des Studiengangs Softwaretechnik und Medieninformatik. \\
Um zu Erfahren wie anspruchsvolle Entwicklungs- und Projektarbeit aussieht, erschien uns dieses Projekt als genau richtig. Es bietet ein relativ offenes Themenfeld, was großen Handlungsspielraum lässt um unsere gewählten Entwicklungsansätze in der Praxis testen zu können. Daraus erhoffen wir uns unsere Fähigkeiten, sei es in der Entwicklung, Planung und generellen Zusammenarbeit mit dem Kunden oder im Team, weiterentwickeln zu können.


\subsection{Rollenverteilung}
Die Rollenverteilung für das Projekt sah dabei wie folgt aus:

%~\\
\begin{center}
	\begin{tabular}{l|l l}
		Kevin Erath & Kunde     & IT-Designers \\
		Öhmer Haybat & Betreuer & IT-Designers \\
		Andreas Rössler & Betreuender Professor & HS-Esslingen \\
		\\
		Oliver Wasser             & Projektmanager und Dokumentation & \\
		Julian Klissenbauer-Mathä & Chefentwickler & \\
		Michael Watzko            & Entwickler, Qualitätsmanager  \\
		                          & und Dokumentation \\
	\end{tabular}
\end{center}

\pagebreak
\section{Umsetzung}

\subsection{Entwicklungsansatz}
Wir haben uns entschieden eine völlig neue Entwicklung zu beginn, da uns die gewählte
Programmiersprache des vorherigen Teams, welches sich mit \textFlowDesign{} beschäftigte,
nicht zugesagt hatte. Mit Java glauben wir Vorteile etwa im Bezug auf 
Plattformunabhängigkeit zu haben, welche mit C\# und dem .NET-Framework nur schwer zu
realisieren wären. \newline
Unsere Implementierung ist demnach in Java geschrieben und benutzt das UI-Toolkit JavaFX
für die Gestaltung der Oberfläche. \newline
Einzig das Konzept des UI wurde von uns grob dem vorherigem Team nachempfunden.

\subsection{Versionsverwaltung}
Für die Versionverwaltung haben wir uns für Git entschieden. Hierbei haben wir auf einen
bereits bestehenden, eigenen Git-Server auf Basis von Gitblit gesetzt. Hierdurch hatten
wir zu jeder Zeit einen Überblick über die Aktivitäten an der Repository und konnten
uns schnell über die letzten Änderungen informieren.

\subsection{Entwicklungsumgebung}
Als Entwicklungsumgebung haben wir IntelliJ IDEA eingesetzt. Nichtsdestotrotz wurde das Projekt
als Maven-Projekt erstellt, wodurch die Maven-Ordnerstruktur als Grundaufbau verwendet wurde.
Auch alle Abhängigkeiten wurden durch Maven aufgelöst. Somit ist es grundsätzlich möglich das
Projekt auch ohne Entwicklungsumgebung zu übersetzen und auszuführen.

\subsection{Lizenz}
Um zukünftigen Teams, die sich mit der gleichen Aufgabe beschäftigen, eine Grundlage zu geben
haben wir uns dazu entschlossen den Quelltext unter der GPL Lizenz zu veröffentlichen. Zudem
wird der Code jederzeit online abrufbar sein.