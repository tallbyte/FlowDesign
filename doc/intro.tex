
\section{Einleitung}

Für das Konstruieren benötigt jeder Ingenieur ein strukturiertes Vorgehen. \textFlowDesign{}
soll genau dort unterstützend wirken.
Zusammen mit der Firma IT-Designers GmbH hat sich unser Team im Wintersemester 2016/17 im
Rahmen des Projekts Softwaretechnik an die Entwicklung eins Programms, dass diesen Ansatz
nutzt, gewagt.

\subsection{Rollenverteilung}
Die Rollenverteilung für das Projekt sah dabei wie folgt aus:

%~\\
\begin{center}
	\begin{tabular}{l|l l}
		Kevin Erath & Kunde     & IT-Designers \\
		Öhmer Haybat & Betreuer & IT-Designers \\
		Andreas Rössler & Betreuender Professor & HS-Esslingen \\
		\\
		Oliver Wasser             & Projektmanager und Dokumentation & \\
		Julian Klissenbauer-Mathä & Chefentwickler & \\
		Michael Watzko            & Entwickler, Qualitätsmanager  \\
		                          & und Dokumentation \\
	\end{tabular}
\end{center}


\subsection{Entwicklungsansatz}
Wir haben uns entschieden eine völlig neue Entwicklung zu beginn, da uns die gewählte
Programmiersprache des vorherigen Teams, welches sich mit \textFlowDesign{} beschäftigte,
nicht zugesagt hatte. Mit Java glauben wir Vorteile etwa im Bezug auf 
Plattformunabhängigkeit zu haben, welche mit C\# und dem .NET-Framework nur schwer zu
realisieren wären. \newline
Unsere Implementierung ist demnach in Java geschrieben und benutzt das UI-Toolkit JavaFX
für die Gestaltung der Oberfläche. \newline
Einzig das Konzept des UI wurde von uns grob dem vorherigem Team nachempfunden.