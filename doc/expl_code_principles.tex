
\section{Einhaltung Programmierkonzepte}

Programmierprinzipien sind essentiell für die Wartbarkeit, Korrektheit und vor allem die Verständlichkeit eines Programmcodes. Die wachsende Komplexität von Programmen und der damit steigende Entwicklungsaufwand erfordern sauberes und strukturiertes arbeiten mehr den je. 


[[ Jeweils mit Code-Beispiel ? ]]

\subsection{SOLID}


\subsubsection{\textPrincipleSingleResponsibility}
\label{\textPrincipleSingleResponsibility}
Das Single-Responsibility-Prinzip besagt, dass eine Klasse, Methode oder Funktion
nur einer Aufgabe verschrieben sein soll. So soll die Funktion zu Berechnung einer
Potenz nur dies tun und nicht gleichzeitig das Ergebnis zbsp auf einer GUI ausgeben.
Dadurch entwickelt sich ein Art Baukastenprinzip bei dem die unterschiedlichen 
Funktionen einfach miteinander verknüpft und oder ausgetauscht werden können, da
keine unnötigen Abhängigkeiten aufgebaut werden.

\subsubsection{\textPrincipleOpenClosed}
\label{\textPrincipleOpenClosed}
Das Open-Closed-Prinzip besagt, dass eine Klasse, Methode oder Funktion offen
für Erweiterungen aber geschlossen für Veränderungen sein soll. So soll es möglich
sein die Klasse Drucker um ein weiteres Papierformat zu erweitern, jedoch soll sich
das Verhalten von verschiedenen von Drucker abgeleiteten Klassen nicht unterscheiden.

\subsubsection{\textPrincipleLiskovSubstitution}
\label{\textPrincipleLiskovSubstitution}
Das Liskovsche Substitutionsprinzip besagt, dass ein Programm so geschrieben sein soll,
dass eine Instanz auf eine Basisklasse durch eine Instanz einer von der Basisklasse
abgeleiteten Klasse ausgetauscht werden können soll. Dadurch wird auch das
\refLongP{\textPrincipleOpenClosed} impliziert.


\subsubsection{\textPrincipleInterfaceSegregation}
\label{\textPrincipleInterfaceSegregation}

\subsubsection{\textPrincipleDependencyInversion}
\label{\textPrincipleDependencyInversion}


\subsection{// TODO KISS}