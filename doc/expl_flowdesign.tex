
\section{\textFlowDesign}

Bei \textFlowDesign{} handelt es sich um eine Methodik, welche dem Entwickler helfen soll
im Voraus saubere Software zu entwerfen. Dabei hilft \textFlowDesign{} etwa bei der
Vermeidung von Abhängigkeiten innerhalb eines Programms durch Entkopplung in einzelne
Module, was ansonsten zu Problemen führen kann.
Datenflüsse werden betrachtet und passende Schnittstellen definiert.

\subsection{System-Umwelt Diagramm}
Beim System-Umwelt Diagramm handelt es sich um eine Betrachtung eines gegebenen oder geplanten Systems, in welcher Interaktionen mit anderen Systemen, Ressourcen und/oder Aktoren dargestellt werden können. Ergänzend hierzu wurde vom Kunden gewünscht, dass bei Interaktionen genauere Details in Listenform hinzugefügt werden können.
Wie im \refLongP{\textMeetingFirst} aufgezeichnet, sollen die einzelnen Interaktionen in der Liste auch mit den passenden Stellen der anderen Diagrammtypen verlinkt sein.


\subsection{Maskenprototyp}
Der Maskenprototyp ist besonders wichtig, um eine grobe Vorstellung zu geben wie das fertig Programm in etwa auszusehen hat. Dieser wird idealerweise in interaktiver Zusammenarbeit, etwa in Kundengesprächen, erstellt. Dabei wurde von unserem Kunden gewünscht, dass bewusst nur rudimentäre Oberflächen erstellbar sind. Damit soll verhindert werden den Eindruck zu vermitteln, dass das Programm bereits fertiggestellt sei.

\subsection{Flow Diagram}